\documentclass[10pt]{article}
\usepackage{amsmath}
\usepackage{array,enumitem,layouts,calc,diagbox}
\usepackage[left=2.2cm,right=2.2cm,top=2.0cm,bottom=2.0cm]{geometry}
\setlist{nosep,leftmargin=2em,rightmargin=0em}
%\usepackage{mathptmx}
\usepackage[T1]{fontenc}
%\renewcommand*\familydefault{\sfdefault} %% Only if the base font of the document is to be sans serif
\newcommand\leftColumn{0.75\textwidth-2\tabcolsep}
\newcommand\rightColumn{0.25\textwidth-2\tabcolsep}
\newcommand\singleColumn{\textwidth-2\tabcolsep}
\newcommand{\ket}[1]{ $\left| #1 \right>$ }
\setlength{\tabcolsep}{0pt}

\begin{document}
\noindent{\large \textbf{30-Aug-2020}: Welcome Video, and What is Machine Learning}
\begin{itemize}
  \item Machine learning: algorithms; supervised, unsupervised, reinforcement, recommender. In this course, also will learn best practices.
\end{itemize}
\hfill \\
{\large \textbf{31-Aug-2020}: Supervised Learning, and Unsupervised Learning}
\begin{itemize}
  \item Supervised learning: right answers are given
  \item Regression: predicts continuous variable output; Classification: predicts discrete values
  \item Classification can have $1,\dots,N,\dots,\infty$ attributes. E.g. benignness/malignancy based on age, or age and tumor size, etc.
  \item Unsupervised learning a.k.a. clustering: Right answers aren't given. For example, news that links to different sources for the same topic.
  \item Cocktail party algorithm: separates two voices in a conversation, with two microphone recordings. Singular value decomposition is key to this algorithm.
  \item When learning machine learning, use Octave
\end{itemize}
\hfill \\
{\large \textbf{1-Sep-2020}: Model Representation, and Cost Function}
\begin{itemize}
  \item Training set notation: $m$ is number of training examples, $x$ are input examples, and $y$ are the output variables. Together, $(x,y)$ form a training example. Also denoted $(x^{(i)},y^{(i)})$.
  \item In a linear regression, $h_{\theta}(x) = \theta_0 + \theta_1 x \equiv h(x)$.
  \item Cost function is 
    \begin{equation*}
      J(\theta_0, \theta_1) = \frac{1}{2m} \sum_{i=1}^m \left( h_{\theta}(x^{(i)}) - y^{(i)} \right)^2
    \end{equation*}
  \item Want to minimize $J$ w.r.t. $\theta_0$ and $\theta_1$.
\end{itemize}
{\large \textbf{4-Sep-2020}: Cost Function, Intuition I\&II; Gradient Descent}
\begin{itemize}
  \item Intuition I; Let $\theta_0=0$, then $\min_{\theta_1}J(\theta_1)$ is what we want
  \item Ex: $h_{\theta}(x)=\theta_1x$ and let $(x,y) = \{ (1,1), (2,2), (3,3) \}$.
    \begin{align*}
      J(\theta_1) &= \frac{1}{2m} \sum_{i=1}^m \left( h_{\theta}(x^{(i)}) - y^{(i)} \right)^2 \\
      &\rightarrow \text{ If $\theta_1=0$, $h_{\theta}(x) \equiv 0$} \\
      J(0) &= \frac{1}{2 \times 3} \left( 1 + 4+9\right) \\ 
      &= \frac{14}{6}
    \end{align*}
  \item $J(\theta_1)$ is parabolic
  \item We want $\min_{\theta}J(\theta)$; here, $\theta_1=1$ satisfies this criterion
  \item Intuition II; Let $\theta_0, \theta_1$ be free in $J(\theta_0, \theta_1)$ and $h_{\theta}(x)$.
  \item $J(\theta_0, \theta_1)$ is a parabloid
  \item Gradient Descent; Use gradient descent to find $(\theta_0, \theta_1)$ that minimizes $J(\theta_0, \theta_1)$.
  \item Differing starting guesses can give different local minima.
  \item Gradient descent algorithm:
    \begin{equation*}
      \theta_j := \theta_j - \alpha \frac{\partial}{\partial \theta_j} J(\theta_0, \theta_1) \quad \text{for $j=1,2$}
    \end{equation*}
  \item Simultaneously update $\theta_0, \theta_1$, $\alpha$ is called the learning rate.
  \item Ex: $\theta_0=1, \theta_2=2$ and $\theta_j := \theta_j + \sqrt{\theta_0 \theta_1}$.
    \begin{align*}
      \theta_0 &:= \theta_0 + \sqrt{\theta_0 \theta_1} \\
      &= 1 + \sqrt{1 \times 2} \\
      &= 1 + \sqrt{2} \\
      \theta_1 &= \theta_2 + \sqrt{ \theta_0 \theta_1 } \\
      &= 2 + \sqrt{1 \times 2} \quad \text{note here that we used the old value of $\theta_0$} \\
      &= 2 + \sqrt{2}
    \end{align*}
\end{itemize}
{\large \textbf{5-Sep-2020}: Gradient Descent Intuition, Gradient Descent for Linear Regression}
\begin{itemize}
  \item Gradient Descent Intuition: For simplicity, assume $\theta_0=0$
  \item One variable: $\theta_1 := \theta_1 - \alpha \frac{d}{d \theta_1}J(\theta_1)$; Newton-Raphson
  \item If $\alpha$ is too small, convergence may be very slow. If too large, it may miss the minimum.
  \item If $\theta_1$ is already at a local minimum, g.d. leaves $\theta_1$ unchanged since the derivative is zero.
  \item Gradient Descent for Linear Regression: We need derivatives
    \begin{align*}
      \frac{ \partial }{\partial \theta_0} J(\theta_0, \theta_1) &= \frac{1}{m} \sum_{i=1}^m \left( \theta_0 + \theta_1 x^{(i)} - y^{(i)}\right) \\
      \frac{ \partial }{\partial \theta_1} J(\theta_0, \theta_1) &= \frac{1}{m} \sum_{i=1}^m \left( \theta_0 + \theta_1 x^{(i)} - y^{(i)}\right) \times x^{(i)}
    \end{align*}
  \item So, gradient descent finds the new $\theta$ variables as
    \begin{align*}
      \theta_0 &:= \theta_0 - \alpha \frac{1}{m} \sum_{i=1}^m \left( \theta_0 + \theta_1x^{(i)} - y^{(i)} \right)\\
      \theta_1 &:= \theta_1 - \alpha \frac{1}{m} \sum_{i=1}^m \left( \theta_0 + \theta_1x^{(i)} - y^{(i)} \right) \times x^{(i)}
    \end{align*}
  \item This is called ``batch gradient descent''; batch implies looking at all the training examples. This is represented by the $\sum_{i=1}^m$.
  \item Quiz Linear Regression with One Variable: 2) $m= \Delta y/ \Delta x = (1-0.5)/(2-1) = 0.5 \implies y=0.5x + b$; y-intercept is clearly zero since (0,0) is a data point.
  \item 3) $h_{\theta}(x)$; $\theta_0=-1$, $\theta_1=2$; $h_{\theta}(6) = -1 + 2 \times 6 = 11$
\end{itemize}
\hfill \\
{\large \textbf{9-Sep-2020}: Linear Algebra Review}
\begin{itemize}
  \item Matrices and Vectors: Nothing new; in this course, index from 1.
  \item Addition and Scalar Multiplication: Nothing new
  \item Matrix Vector Multiplication: Nothing new;
  \item Ex: Let house sizes be $\{2104, 1416, 1534, 852.\}$. Let the hypothesis be $h_{\theta}(x) = -40 + 0.25x$.
    \begin{equation*}
      \begin{bmatrix}
        1 & 2104 \\
        1 & 1416 \\
        1 & 1534 \\
        2 & 852 
      \end{bmatrix}
      \begin{bmatrix}
        -40 \\ 
        0.25
      \end{bmatrix} 
      =
      \begin{bmatrix}
        -40 \times 1 + 0.25 \times 2104 \\
        -40 \times 1 + 0.25 \times 1416 \\
        -40 \times 1 + 0.25 \times 1534 \\
        -40 \times 1 + 0.25 \times 852
      \end{bmatrix}
      =
      \begin{bmatrix}
        h_{\theta}(2104)\\
        h_{\theta}(1416)\\
        h_{\theta}(1534)\\
        h_{\theta}(852)
      \end{bmatrix}
    \end{equation*}
    This essentially says data matrix $\times$ parameters = prediction
  \item Best to do this with built-in linear algebra function in Octave/Python. You can do it manually in a for-loop, but it'll be really slow.
  \item Matrix Multiplication: Take the same example. Now we have three hypotheses:
    \begin{align*}
      h_{\theta}(x) &= -40 + 0.25x \\
      h_{\theta}(x) &= 200 + 0.1x \\
      h_{\theta}(x) &= -150 + 0.4x 
    \end{align*}
    In matrix form, this becomes
    \begin{equation*}
      \begin{bmatrix}
        1 & 2104 \\
        1 & 1416 \\
        1 & 1534 \\
        2 & 852 
      \end{bmatrix}
      \begin{bmatrix}
        -40 & 200 & -150 \\
        0.25 & 0.1 & 0.4
      \end{bmatrix}
      =
      \begin{bmatrix}
        486 & 410 & 692 \\
        314 & 342 & 416 \\
        344 & 353 & 464 \\
        173 & 285 & 191
      \end{bmatrix}
    \end{equation*}
  \item Matrix Multiplication Properties: Not commutative. $AB \neq BA$. But it's associative. $ABC = (AB)C = A(BC)$.
  \item Identity matrix is $I$ such that $AI = IA = A$. $I = \begin{bmatrix}1 & 0 \\ 0 & 1\end{bmatrix}$ in 2D.
  \item Inverse of $A$ is $A^{-1}$ such that $AA^{-1} = A^{-1}A = I$.
  \item Transpose of $A$ is $A^{T}$. If $B=A^{T}$, then $B_{ij} = A_{ji}$.
  \item Quiz: 4) $u = \begin{bmatrix}3 \\ -5 \\ 4\end{bmatrix}, v= \begin{bmatrix}1 \\ 2 \\ 5\end{bmatrix}$, then $u^Tv = \begin{bmatrix}3 & -5 & 4\end{bmatrix}\begin{bmatrix}1 \\ 2\\ 5 \end{bmatrix} = -3 + (-10) + 20 = 13$.
\end{itemize}
\hfill \\
{\large \textbf{10-Sep-2020}: Multiple Features}
\begin{itemize}
  \item Introduce other features: e.g. house price not just a function of square footage; Now, house price vs. sq. footage, age, number of bedrooms, etc.
  \item $n$ is the number of features, $x_j^{(i)}$ represents the value of the $j^{th}$ feature for the $i^{th}$ training example; $x^{(i)}$ is a vector of all the features.
  \item Hypothesis: $h_{\theta}(x) = \theta_0 + \theta_1x_1 + \theta_2 x_2 + \dots + \theta_n x_n$. Let $x_0^{(i)}=1$. Then, we can write this in matrix form as
    \begin{equation*}
      x=\begin{bmatrix}x_0 \\ x_1 \\ \vdots \\ x_n\end{bmatrix}, \quad \theta = \begin{bmatrix} \theta_0 \\ \theta_1 \\ \vdots \\ \theta_n \end{bmatrix} \implies h_{\theta}(x) = \theta^T x
    \end{equation*}
\end{itemize}
\hfill \\
{\large \textbf{11-Sep-2020}: G.D. for Multiple Variables, G.D. in Practice I - Feature Scaling, G.D. in Practice II - Learning Rate, Features and Polynomial Regression, Normal Equation}
\begin{itemize}
  \item Gradient Descent for Multiple Variables: For $n \geq 1$, gradient descent is
    \begin{equation*}
      \theta_j := \theta_j - \alpha \frac{1}{m} \sum_{i=1}^m \left( h_{\theta}(x^{(i)}) - y^{(i)}\right)x_j^{(i)} \quad \text{for $j=0,\dots,n$.}
    \end{equation*}
  \item G.D. in Practice I - Feature Scaling: ensure features have
    similar scales. E.g.: Houses in the data set have 1-5 bedrooms,
    and are between 0-2000 sq. ft. Scale these features to the order
    of 1. So, divide bedrooms by 5 so it's 0-1, and divide square
    footage by 2000, so it's 0-2.
  \item Feature should be $-1 \leq x_i \leq 1$.
  \item Mean renormalization; Subtract off the mean, and then scale. E.g. $x_1= (\text{sq. footage} - 1000)/2000$ and $x_2 = (\text{bedrooms} - 2)/5$. More formally,
    \begin{equation*}
      x_i \rightarrow \frac{x_i - \mu_i}{s_i} \quad \text{(mean renormalization),}
    \end{equation*}
    where $x_i$ is the feature, $\mu_i$ is the mean value of the
    $i^{th}$ feature, and $s_i$ is the range, or standard deviation,
    of the $i^{th}$ feature.
  \item G.D. in Practice II - Learning Rate: We can plot $J(\theta)$ as a function of iterations, $N$; it should be a decreasing function.
  \item If $J(\theta)$ vs $N$ diverges, you need a smaller learning rate, $\alpha$.
  \item If $J(\theta)$ vs $N$ falls, rises, falls, rises, etc., then use a smaller $\alpha$.
  \item Features and Polynomial Regression: In the housing example,
    hypothesis could be $h_{\theta}(x) = \theta_0 + \theta_1 \times
    \text{length} + \theta_2 \times \text{depth}$.  Maybe you think
    the relevant figure is area = length$\times$depth$\equiv x$. The
    hypothesis is $h_{\theta}(x) = \theta_0 + \theta_1 \times x$.
  \item Polynomial regression; e.g. 
    \begin{alignat*}{4}
      &\theta_0 + \theta_1&&x   + \theta_2&&x^2 + \theta_3&&x^3 \\
      &\theta_0 + \theta_1&&x_1 + \theta_2&&x_2 + \theta_3&&x_3 \\
    \end{alignat*}
    where $x_1=x=\text{area}$, $x_2=x^2 = \text{area}^2$, $x_3=x^3=\text{area}^3$. In polynomial regression, feature scaling becomes very important.
  \item Don't just have to have integer powers: e.g. $h_{\theta}(x) = \theta_0 + \theta_1x + \theta_2 x^{1/2}$
  \item Normal Equation: Instead of using gradient descent to find $\min_{\theta}J$, use normal equation to do it analytically.
  \item Intuition; in 1-D, if $J(\theta)=a\theta^2 + b\theta+c$, you can find $dJ/d\theta=0$ to get the extremum. In $N$-D, set $\partial_{\theta_j}J=0$ for $j=1,\dots,N$.
  \item Say you have $m$ training examples, each with $n$ features. Let
    \begin{align*}
      X_{ij} &= x_j^{(i)} \\
      Y_i &= y^{(i)} \\
      \theta &= \left( X^T X \right)^{-1} X^T Y
    \end{align*}
  \item If the training examples are $\left( x^{(1)}, y^{(1)} \right), \dots, \left( x^{(m)}, y^{(m)} \right)$, then
    \begin{equation*}
      x^{(i)} = \begin{bmatrix}x_0^{(i)} \\ \vdots \\ x_n^{(i)}\end{bmatrix}, 
      X = \begin{bmatrix} \left.\vec{x}^{(1)}\right.^T \\ \vdots \\ \left.\vec{x}^{(m)}\right.^T \end{bmatrix},
      Y = \begin{bmatrix} y^{(1)} \\ \vdots \\ y^{(m)} \end{bmatrix},
      \theta = \left( X^T X\right)^{-1} X^T Y,
    \end{equation*}
    where $x_0^{(i)}=1$.
  \item With normal equation method, features don't have to be scaled.
  \item Normal equation method is slow if $n$ is very large; Computing $(X^TX)^{-1}$ is costly. Inverting an $N \times N$ matrix costs $O(N^3)$.
\end{itemize}
\hfill \\
{\large \textbf{12-Sep-2020}: Normal Equation and Non-Invertibility}
\begin{itemize}
  \item What if $X^TX$ is signular? Octave's \texttt{pinv} (pseudo-inverse) takes care of that
  \item Causes; redundancy: e.g. area in ft$^2$ and in m$^2$; too few equations: $m<n$, fewer training examples compared to features, i.e. too few equations, too many unknowns.
  \item Quiz: 1) Midterm exam average $\mu_1 = (89+72+94-69)/4=81$; range is $s_1= 94-69=25$, thus $x_1^{(3)} \rightarrow (x^{(3)}-\mu_1)/s_1 = (94-81)/25=0.52$
  \item 3) $X = \begin{bmatrix}x_0^{(1)} & \dots & x_3^{(1)} \\ \hfill & \vdots & \hfill \\ x_0^{(14)} & \dots & x_3^{(14)} \end{bmatrix}$ is 14$\times$4.
\end{itemize}
\hfill \\ \hfill \\
{\large \textbf{13-Sep-2020}: Octave Quiz}
\begin{itemize}
  \item Quiz: 1) A is 3$\times$2, B is 2$\times$3. Thus, AB and A+B$^T$ are valid
  \item 4) $u,v$ are $7\times1$. Calculate $u \cdot v$. This can be done via $u^Tv$. In Octave, this is \texttt{sum(v.*w)} and \texttt{v'*w}
\end{itemize}
\hfill \\
{\large \textbf{17-Sep-2020}: Classification, Hypothesis Representation, Decision Boundary, Cost Function, Simplified Cost Function and Gradient Descent}
\begin{itemize}
  \item Classification: $y \in \{ 0,1 \}$ (binary), $y \in \{ 0,1,2, \dots, N \}$ (multiclass)
  \item Could fit a linear $h_{\theta}(x) = \theta^Tx$ and classify
    using a threshold of 0.5. Not good, though. Too sensitive to
    outliers. Also, $h_{\theta}(x)$ can be negative.
  \item Hypothesis Representation:
    \begin{align*}
      h_{\theta}(x) &= g(\theta^Tx); \\
      g(z) &= \frac{1}{1+e^{-z}} \quad \text{sigmoid or logistic function; Fermi-Dirac distribution;}\\
      &\quad \implies h_{\theta}(x) = \frac{1}{1 + \exp\left( -\theta^T x\right)}.
    \end{align*}
  \item $h_{\theta}(x)$ represents the probability that $y=1$ on an
    input x. E.g. $\begin{bmatrix}x_0 \\ x_1 \end{bmatrix}
    = \begin{bmatrix}1 \\ \text{tumor size} \end{bmatrix}$. If
    $g(x)=70\%$, then there is a 70\% chance that the tumor is
    malignant.
  \item $h_{\theta}(x) = P \left( \left. y=1 \right| x; \theta \right)$ means ``the probability that $y=1$ given x, parametrized by $\theta$''
  \item Probabilities sum to 1. $P \left( \left. y=1 \right| x; \theta \right) + P \left( \left. y=0 \right| x; \theta \right) = 1$.
  \item Decision Boundary: Can say if $h_{\theta}(x) \geq 0.5 \implies y=1$, $h_{\theta}(x) < 0.5 \implies y=0$.
  \item $g(z) \geq 0.5 \implies z \geq 0$. So, $h_{\theta}(x)=g(\theta^Tx) \geq 0.5 \implies \theta^Tx \geq 0.$ Converse is true for $<0.5$.
  \item Decision Boundary; Say $g(\theta_0 + \theta_1x_1 +
    \theta_2x_2)$. WLOG, let $\theta_1=\theta_2=1$. On an $x_2$-$x_1$
    diagram, this parametrizes a straight line. $x_2 = -x_1 +
    \theta_0$. The decision boundary is the set of points $(x_2,x_1)$
    s.t. $h_{\theta}(x)=0.5$.
  \item Example; $\theta_0=5, \theta_1=-1, \theta_2=0 \implies h_{\theta}(x) = g(5-x_1)$. Decision boundary is implied by $x_1=5$. Where is $h \geq 0.5$? $5-x \geq 0 \implies x_1 \leq 5$. This region corresponds to $y=1$.
  \item Non-Linear Decision Boundaries; $g(\theta_0 + \theta_1x_2 + \theta_2x_2 + \theta_3x_1^2 + \theta_4x_2^2)$.
  \item Cost Function: Training set $\{ (x^{(1)}, y^{(1)}), \dots, (x^{(m)},y^{(m)})\}$, $x^{(i)} = \begin{bmatrix}x_0^{(i)} \\ \vdots \\ x^{(i)}_n\end{bmatrix}$. $m$ training examples, each with $n$ features. $x_0^{(i)} \equiv 1$.
  \item Recall, for linear regression, $J(\theta) = \frac{1}{m} \sum_{i=1}^m \frac{1}{2} \left( h_{\theta}(x^{(i)}) - y^{(i)} \right)^2$. Define
    \begin{equation*}
      \text{cost}(h_{\theta}(x),y) = \frac{1}{2} \left( h_{\theta}(x) - y\right)^2, \quad \text{for linear regression only}
    \end{equation*}
    so $J(\theta) = \frac{1}{m}\sum_{i=1}^m \text{cost}(h_{\theta}(x^{(i)}, y^{(i)})$.
  \item For log. regression, $J(\theta)$ is not convex, i.e. it has many local minima. Need a new cost function.
  \item For log. regression, $\text{cost}(h_{\theta}(x),y) = \begin{cases} -\log(h_{\theta}(x)) & \mbox{if } y=1 \\ -\log(1-h_{\theta}(x)) & \mbox{if } y=0 \end{cases}$
  \item If $y=1$, and $h_{\theta}(x)=1$, the cost=0. As $h_{\theta}(x)
    \rightarrow 0$, $\text{cost} \rightarrow \infty$. Converse is also
    true for $y=0$.
  \item Simplified Cost Function For Gradient Descent: Write out cost function in one equation
    \begin{equation*}
      \text{cost}(h_{\theta}(x),y) = -y \log(h_{\theta}(x)) - (1-y) \log (1 - h_{\theta}(x))
    \end{equation*}
    Plug into $J$.
    \begin{equation*}
      J(\theta) = \frac{1}{m} \sum_{i=1}^m \left( -y^{(i)} \log\left(h_{\theta}(x^{(i)})\right) - (1-y^{(i)}) \log \left(1- h_{\theta}(x) \right) \right)
    \end{equation*}
    Comes from ``max likelihood estimation.'' Then find $\theta$ via $\min_{\theta}J(\theta)$. USe $\theta$ to make predictions $h_{\theta}(x) = 1/(1 + \exp(\theta^Tx))$.
  \item Use g.d.: $\theta_j := \theta_j - \alpha \partial_{\theta_j} J(\theta)$. Computing the derivatives, we have
    \begin{equation*}
      \theta_j := \theta_j - \alpha \sum_{i=1}^m \left( h_{\theta}(x^{(i)}) - y^{(i)}\right)x_j^{(i)},
    \end{equation*}
    which is the same as for linear regression, though $h$ has a different meaning.
\end{itemize}
\hfill \\
{\large \textbf{2-Oct-2020} Advanced Optimization, Multiclass Classification One-Vs.-All, Problem of Overfitting, Cost Function (Regularization), Reg. Lin. Regr., Reg. Log. Regr.}
\begin{itemize}
  \item Advanced Optimization; nothing new
  \item Multiclass Classification: One-vs.-all; e.g. weather: sunny, rainy, snowy, etc. $h_{\theta}^{(i)}(x) = P \left( \left. y=i \right| x; \theta \right)$ represents the boundary separating class $i$ from the rest.
  \item Quiz; $h_{\theta}(x) = g(\theta_0 + \theta_1x + \theta_2x)$. Let $\theta_0=6$, $\theta_1=-1$, $\theta_2=0$. The argument, $z$, of $g$ is positive.  $z=0 \implies x=6$, and $z \geq 0 \implies x \leq 6$. So, for $y=1$, $x \leq 6$.
  \item Problem of Overfitting; Underfit means high bias, overfit means high variance. Overfitting fails to generalize to new examples. To fix overfitting you can (i) reduce the number of examples, (ii) regularize to reduce the magnitude of the $\theta_i$.
  \item Cost Function (Regularization); Add terms to the cost function such as $1000\theta_3^2$. This will force $\theta_3$ down.
  \item But what if we don't know what features we want to be small? Do
    \begin{equation*}
      J(\theta) = \frac{1}{2m} \left[ \sum_{i=1}^m \left( h_{\theta}(x^{(i)}) - y^{(i)}\right)^2  + \underbrace{\lambda \sum_{i=1}^m \theta_j^2}_{\text{Regularization term}} \right],
    \end{equation*}
    where we do not penalize $\theta_0$. If $\lambda$ is too large, it underfits.
  \item Regularized Linear Regression; G.D. for lin. regr.:
    \begin{align*}
      \theta_j &:= \theta_j - \alpha \left[ \frac{1}{m} \sum_{i=1}^m \left( h_{\theta}(x^{(i)}) - y^{(i)} \right)x^{(i)} + \frac{\lambda}{m}\theta_j \right] \\
      &:= \theta_j \left( 1 - \alpha \frac{\lambda}{m} \right) - \alpha \frac{1}{m} \sum_{i=1}^m \left( h_{\theta}(x^{(i)}) - y^{(i)} \right)x^{(i)},
    \end{align*}
    where $1 - \alpha \lambda/m < 1$ which reduces $\theta_j$.
  \item Normal equation becomes 
    \begin{equation*}
      \theta = \left(x^Tx + \lambda \begin{bmatrix} 0 & & & \\ & 1 & & \\ & & \ddots & \\ & & & 1 \end{bmatrix}\right)^{-1}x^Ty,
    \end{equation*}
    and as long as $\lambda>0$, the matrix will not be singular.
  \item Regularized Log. Regression; Cost function is
    \begin{equation*}
      J(\theta) = - \frac{1}{m} \sum_{i=1}^m \left[y^{(i)} \log h_{\theta}(x^{(i)}) + (1 -y^{(i)})\log \left( 1 - h_{\theta}(x^{(i)}) \right)\right] + \frac{\lambda}{2m} \sum_{i=1}^m \theta_j^2
    \end{equation*}
  \item G.D. becomes (same cosmetically as for lin. regr.)
    \begin{equation*}
      \theta_j \left( 1 - \alpha \frac{\lambda}{m} \right) - \alpha \frac{1}{m} \sum_{i=1}^m \left( h_{\theta}(x^{(i)}) - y^{(i)} \right)x^{(i)}.
    \end{equation*}
  \item Use \texttt{fminunc} (unc means unconstrained). For this, you need to give derivatives.
    \begin{align*}
      \frac{ \partial }{\partial \theta_0} J(\theta) &= \frac{1}{m} \sum_{i=1}^m \left( h_{\theta}(x^{(i)}) - y^{(i)} \right)x_0^{(i)} \\
      \frac{ \partial }{\partial \theta_j} J(\theta) &= \frac{1}{m} \sum_{i=1}^m \left( h_{\theta}(x^{(i)}) - y^{(i)} \right)x_j^{(i)} + \frac{\lambda}{m} \theta_j
    \end{align*}
\end{itemize}
\hfill \\
{\large \textbf{6-Oct-2020} Neural Networks: Non-Linear Hypothesis, Neurons and the Brain}
\begin{itemize}
  \item e.g. identifying a car. Come up with a classification problem. Take images of a car and select two pixels. Take their intensity, and form an ordered pair $(I(p_1),I(p_2))$. Do this for cars and non-cars. Cars and non-cars will lie in different regions.
  \item If images were 50x50 pixels, $n=2500$. (2500 pixels per image). This would make $3\times10^6$ quadratic features. So, log regression with quadratic features doesn't work.
\end{itemize}
\hfill \\
{\large \textbf{7-Oct-2020} Model Representation I and II, Examples and Intutitions I and II}
\begin{itemize}
  \item Model Representation I; A neuron is a logistic unit. Input is $\{x_1,\dots,x_n\} \rightarrow h_{\theta}(x) = g(\theta^Tx)$ where $g(z) = (1+\exp(-z))^{-1}$. Sigmoid activation function.
  \item We call $x_0$ the bias unit, and in neural networks, $\theta$ are called the weights. We call the inputs the ``input layer'', the last layer of neurons the ``output layer,'' and the rest ``hidden layers.''
  \item $a_i^{(j)}$ is the activation of unit $i$ in the layer $j$. $\Theta^{(j)}$ is the matrix of weights controlling the function mapping from layer $j$ to layer $j+1$.
  \item cf. p. 20 of lecture 08.pdf for a detailed picture.
  \item If a network has $s_j$ units in layer $j$ and $s_{j+1}$ units in layer $j+1$, then $\Theta^{(j)}$ is of dimension $s_{j+1} \times (s_j + 1)$.
  \item Model Representation II; Define $z_1^{(2)} = \Theta^{(1)}_{10}x_0 + \Theta^{(1)}_{11}x_1 + \Theta^{(1)}_{12}x_2 + \Theta^{(1)}_{13}x_3$ such that $a_1^{(s)} = g(z_1^{(2)})$ \hfill \\
    Then let $z^{(2)} = \begin{bmatrix}z_1^{(2)} \\ z_2^{(2)} \\ z_3^{(2)} \end{bmatrix}$, $x = \begin{bmatrix} x_0 \\ x_1 \\ x_2 \\ x_3 \end{bmatrix}$. Then, $z^{(2)} = \Theta^{(1)}x$ and $a^{(2)} = g(z^{(2)})$. (These last two are vectors, and $g$ is applied element-wise. Here $\Theta^{(1)}$ is $3 \times 4$)
  \item cf. p. 23 of lecture 08.pdf. We may also call the input layer $a^{(1)}$. To add a bias for $a^{(3)}$, define $a_0^{(2)}=1$. Thus $a^{(2)} \in \mathcal{R}^4$. 
  \item This is forward propagation. In the example, on p. 23, imagine hiding the input layer. Then we have $a_1^{(2)}$, $a_2^{(2)}$, $a_3^{(2)}$ feeding to $a^{(3)}$.
    \begin{align*}
      a^{(3)} &= g( \Theta^{(2)}_{10}x_0 + \Theta^{(2)}_{11}x_1 + \Theta^{(2)}_{12}x_2 + \Theta^{(2)}_{13}x_3 \\
      &= g( \theta_0x_0 + \theta_1x_1 + \theta_2x_2 + \theta_3x_3) \quad \text{drop some super/subscripts}\\
      &= g( \theta^T x) \rightarrow \text{ just like log. regression.}
    \end{align*}
    The difference comes in that it's not the straight inputs passed to the log regression. It's the ``learned'' features of the first layer.
  \item An architecture refers to how the networks are connected.
  \item Examples and Intutions I; $x_1$ NOR $x_2$ = NOT($x_1$ XOR $x_2$). XOR is True if $x_1$ OR $x_2$ is True.
  \item cf. p. 34 of lecture 08.pdf for a n.n. implementation of AND, OR, and (NOT $x_1$) AND (NOT $x_2$)
  \item For a NOT, use a large negative weight on the neurons other than the bias.
  \item cf. p. 40-42 re: multiclass classification.
\end{itemize}
\hfill \\
{\large \textbf{8-Oct-2020} Cost Function and Back Propagation, Backpropagation Algorithm}
\begin{itemize}
  \item Cost Function and Back Propagation; $L$ = \# of layers, $s_l$ is \# of units in layer $l$, excluding bias.
  \item Binary classification: $s_L=1$, $h_{\theta}(x) \in \mathcal{R}$, $K=1$ (i.e. a real number). $y = \{ 0,1 \}$.\\
    Multiclass classification: $s_L=K$, $h_{\theta}(x) \in \mathcal{R}^K$, $K \geq 3$, $y \in \mathcal{R}^K$ e.g. $\begin{bmatrix}1 \\ 0 \\ 0 \\ 0\end{bmatrix}, \begin{bmatrix}0 \\ 1 \\ 0 \\ 0 \end{bmatrix}$, etc.
  \item For a n.n., $h_{\theta}(x) \in \mathcal{R}^K$, $\left[ h_{\theta}(x)\right]_i = i^{th}$ output. The cost function is
    \begin{align*}
      J( \Theta ) &= -\frac{1}{m} \left[ \sum_{i=1}^m \sum_{k=1}^K y_k^{(i)} \log \left( h_{\Theta}(x^{(i)})\right)_k + (1-y_k^{(i)}) \log \left( 1 - \left( h_{\Theta}(x^{(i)}) \right)_k \right) \right] \\
      &+ \frac{\lambda}{2m} \sum_{l=1}^{L-1} \sum_{i=1}^{s_l} \sum_{j=1}^{s_l+1} \left( \Theta_{ji}^{(l)} \right)^2.
    \end{align*}
  \item Backpropagation Algorithm; Wish to get $\min_{\Theta}J(\Theta)$. Need $J(\Theta)$ and $\frac{\partial}{\partial \Theta_{ij}^{(l)}} J(\Theta)$ for all $i,j,l$.
  \item Suppose we have just one training example $(x,y)$. See p.6 of lecture 09.pdf for forward propagation.
  \item $\delta^{l}_j$ represents the error of the $j^{th}$ unit in the $l^{th}$ layer. \\
    For the output layer, (cf. p. 7), $\delta_j^{(4)} = a_j^{(4)} - y_j = \left[ h_{\theta}(x)\right]_j - y_j$. Vectorize this to have $\delta^{(4)} = a^{(4)} - y$. 
  \item Look at $\delta^{(3)}$.
    \begin{equation*}
      \delta^{(3)} = \left( \Theta^{(3)} \right)^T \delta^{(4)} \underbrace{.*}_{\text{element-wise}} g^{\prime} \left( z^{(3)} \right), 
    \end{equation*}
    where $g^{\prime}$ is the derivative of the activation function. It can be shown that $g^{\prime}(z^{(3)}) = a^{(3)}.*\left(1-a^{(3)}\right)$. There is no $\delta^{(1)}$ since we assume there is no error to the input.
  \item We call this backpropagation since we start at the output layer and get $\delta^{(4)}$, and use it to get $\delta^{(3)}$, etc.
  \item Without regularization, we have 
    \begin{equation*}
      \frac{ \partial }{\partial \Theta_{ij}^{(l)}} J (\Theta) = a_j^{(l)} \delta_i^{(l+1)}
    \end{equation*}
  \item Now say the training set is $\{ (x^{(1)}, y^{(1)}), \dots, (x^{(m)}, y^{(,)})\}$. Let $\Delta_{ij}^{(l)}=0$ for all $i,j,l$. Pseudocode:
  \item For $i$=1 to $m$:
    \begin{itemize}
      \item  $a^{(1)} = x^{(i)}$
      \item Do forwards prop to get $a^{(l)}$ for $l=2,\dots,L$
      \item Use $y^{(i)}$ to get $\delta^{(L)} = a^{(L)} - y^{(i)}$
      \item Use back propagation to get $\delta^{(L-1)},\dots,\delta^{(2)}$
      \item $\Delta_{ij}^{(l)} := \Delta_{ij}^{(l)} + a_j^{(l)}\delta_i^{(l+1)} \rightarrow$ vectorize: $\Delta^{(l)} := \Delta^{(l)} + \delta^{(l+1)}(a^{(l)})^T$
    \end{itemize}
    Then, calculate
    \begin{itemize}
      \item $D_{ij}^{(l)} := \frac{1}{m} \Delta_{ij}^{(l)} + \lambda \Theta_{ij}^{(l)}$ if $j \neq 0$
      \item $D_{ij}^{(l)} := \frac{1}{m} \Delta_{ij}^{(l)}$ if $j =0$
    \end{itemize}
    \begin{equation*}
      \implies \frac{\partial}{\partial \Theta_{ij}^{(l)}} J( \Theta ) = D_{ij}^{(l)}
    \end{equation*}
\end{itemize}
\hfill \\
{\large \textbf{9-Oct-2020}: Backpropagation Intutition, Implementation Note, Gradient Checking}
\begin{itemize}
  \item Backpropagation Intuition; Quiz: Suppose $\Theta_{11}^{(2)} =
    \Theta_{21}^{(2)} = 0$. After back propagation, what can you say
    about $\delta_1^{(3)}$? We know that $\delta_1^{(3)} =
    \Theta_{11}^{(3)} \delta_1^{(4)}$. So, there is insufficient
    information to tell.
  \item Implementation Note; Unrolling Parameters. In Octave, say \texttt{v} is a matrix. Then \texttt{v(:)} makes it a vector.
  \item Gradient Checking; Use symmetric difference. In 1-D
    \begin{equation*}
      \frac{d}{d \theta} J(\theta) = \frac{ J(\theta + \epsilon) - J(\theta - \epsilon) }{2\epsilon}.
    \end{equation*}
  In N-D,
    \begin{equation*}
      \frac{\partial}{\partial \theta_j} J(\theta) = \frac{J( \theta_1, \dots, \theta_j + \epsilon, \dots, \theta_n) - J( \theta_1, \dots, \theta_j - \epsilon, \dots, \theta_n) }{2 \epsilon}
    \end{equation*}
  \item Use this to compare to derivs we get from back-propagation. Don't use it for training! It will be very slow.
\end{itemize}
\hfill \\
{\large \textbf{13-Oct-2020} Random Initialization, Putting it Together, Autonomous Driving}
\begin{itemize}
  \item Random Initialization; Have to give a guess for $\Theta^{(l)}$; can't just guess all 0s. If the initial $\Theta$ are all zero, it means all units are looking at the same features. Initialize $\Theta_{ij}^{(l)}$ to a random $[ -\epsilon, \epsilon ]$.
  \item Putting it Together; Pick some network architecture. Input
    layer will have a size of $x^{(1)}$. Output layer is the number of
    classes. As a default, use one hidden layer. If $>1$ hidden layer,
    then use the same number of units in each hidden layer.
    \item To train a neural network:
      \begin{enumerate}
        \item Randomly initialize the weights
        \item Do forward propagation to get $h_{\Theta}(x^{(i)})$ for any $x^{(i)}$.
        \item Implement code to compute $J( \Theta )$.
        \item Implement back propagation algorithm to get the partial derivatives. \\
          The backpropagation algorithm will loop over the training examples. Get $a^{(l)}$ and $\delta^{(l)}$ for $l=1, \dots, L$.
        \item Use gradient checking to compare partial derivatives from back propagation. Then discard gradient checking.
        \item Use optimization algorithm to minimize $J( \Theta )$.
      \end{enumerate}
    \item Note that $J( \Theta )$ is non-convex. So, the algorithm can get stuck in a local minimum, but this is rare in practice.
    \item Autonomous Driving; nothing new.
\end{itemize}
\hfill \\
{\large \textbf{15-Oct-2020} Deciding What to Try Next, Evaluating Hypotheses, Model Selection \\ and Train/Validation/Test Sets, Diagnosing Bias vs. Variance, Regularization and Bias/Variance}
\begin{itemize}
  \item Deciding What To Try Next; Suppose, for e.g., you've
    implemented regularized linear regression, and it makes
    predictions with large errors. Try (i) get more training e.g.s,
    (ii) reduce the number of features, (iii) try more features, (iv)
    add polynomial features, (v) increase/decrease $\lambda$.
  \item Simple way to rule out solutions: diagnostics. It can take a while to implement, but can ultimately pay off.
  \item Evaluating a Hypothesis; Split the data into a training/test set. Split the data 70/30 randomly.
  \item For lin. regr., analagous to the training set, the test error is $J_{test}(\theta) = 1/(2m_{test}) \sum_{i=1}^m \left( h_{\theta}(x_{test}^{(i)}) - y_{test}^{(i)} \right)^2$.
  \item Analagous $J_{test}$ for log. regr. For log. regr., also have misclassification error.
    \begin{equation*}
      \text{err}(h_{\theta}(x), y) = \begin{cases} 1 & \mbox{if } h_{\theta}(x) \geq 0.5 \mbox{ and } y=0 \mbox{ OR }  \mbox{ if } h_{\theta}(x) < 0.5 \mbox{ and } y=1 \\ 0 & \mbox{otherwise} \end{cases}
    \end{equation*}
    Could then define the test error to be
    \begin{equation*}
      J_{test}(\theta) = \frac{1}{m_{test}} \sum_{i=1}^{m_{test}} \text{err}\left( h_{\theta}(x_{test}^{(i)}) - y_{test}^{(i)} \right) \quad \text{misclassification error}
    \end{equation*}
  \item Model Selection and Train/Validation/Test Sets; How to choose the polynomial degree $d$? \\
    \begin{center}
      \begin{tabular}{ l c c l l }
        $d=1 \quad$ & $h_{\theta}(x) = \theta_0 + \theta_1x \rightarrow$ & $\min_{\theta}J(\theta) \rightarrow $ & $\theta^{(1)} \rightarrow$ & $J_{test}(\theta^{(1)})$ \\
        $d=2 \quad$ & $h_{\theta}(x) = \theta_0 + \theta_1x + \theta_2x^2 \rightarrow$ & $\min_{\theta}J(\theta) \rightarrow$ & $\theta^{(2)} \rightarrow$ & $J_{test}(\theta^{(2)})$ \\
        $\vdots$ & & & & \\
        $d=n \quad$ & $h_{\theta}(x) = \sum_{k=0}^n \theta_k x^k \rightarrow$ & $\min_{\theta}J(\theta) \rightarrow$ & $\theta^{(n)} \rightarrow$ & $J_{test}(\theta^{(n)})$
      \end{tabular}
    \end{center}
    A problem can be that we're essentially fitting $d$ to the training set. E.g. we find that $J_{test}(\theta^{(5)})$ is the lowest, then we've fit $d=5$ to the training set.
  \item Split the data into a training, cross-validation (CV), and test set (60-20-20). Get the training, CV, and test errors. Repeat the process.
    \begin{center}
      \begin{tabular}{ l c c l l }
        $d=1 \quad$ & $h_{\theta}(x) = \theta_0 + \theta_1x \rightarrow$ & $\min_{\theta}J(\theta) \rightarrow $ & $\theta^{(1)} \rightarrow$ & $J_{cv}(\theta^{(1)})$ \\
        $d=2 \quad$ & $h_{\theta}(x) = \theta_0 + \theta_1x + \theta_2x^2 \rightarrow$ & $\min_{\theta}J(\theta) \rightarrow$ & $\theta^{(2)} \rightarrow$ & $J_{cv}(\theta^{(2)})$ \\
        $\vdots$ & & & & \\
        $d=n \quad$ & $h_{\theta}(x) = \sum_{k=0}^n \theta_k x^k \rightarrow$ & $\min_{\theta}J(\theta) \rightarrow$ & $\theta^{(n)} \rightarrow$ & $J_{cv}(\theta^{(n)})$
      \end{tabular}
    \end{center}
    Pick a $d$ such that $J_{cv}$ is the smallest, e.g. $\theta^{(4)}$, and then get $J_{test}(\theta^{(4)})$, and assess it. So, use cross-validation to select the model (here, polynomial degree).
  \item Diagnosing Bias (Underfit) vs. Variance (overfit); cf. lecture 10, slide 17 for a plot of $J_{cv}$, $J_{test}$ vs $d$.
  \item Bias (underfit): $J_{train}(\theta)$ is high; $J_{train} \approx J_{cv}$. \\
    Variance (overfit):  $J_{train}(\theta)$ is low; $J_{train} \ll J_{cv}$. \\
  \item Regularization and Bias/Variance; For linear regression with
    regularization, 
    \begin{equation*}
      J(\theta) = \sum_{i=1}^m\left(h_{\theta}(x^{(i)}) - y^{(i)} \right)^2 + \frac{\lambda}{2m}\sum_{j=1}^m \theta_j^2. 
    \end{equation*}
    Suppose we're using a fourth-order polynomial, so $h_{\theta}(x) =
    \sum_{k=0}^4 \theta_kx^k$.\\ For very large $\lambda$, $\theta_1,
    \dots, \theta_4 \approx 0$ (heavily penalized) $\rightarrow$ underfit
    $\rightarrow$ high bias. \\ For very small $\lambda$, $\theta_1,
    \dots, \theta_4 \neq 0$ (very little regularization) $rightarrow$
    overfit $\rightarrow$ high variance.\\ We need a goldilocks $\lambda$.
  \item With regularization, change definitions a bit.
    \begin{align*}
      J_{train}(\theta) &= \frac{1}{2m}\sum_{i=1}^m \left( h_{\theta}(x^{(i)}) - y^{(i)} \right)^2 \quad \text{(NO regularization terms)} \\
      J_{cv}(\theta) &= \frac{1}{2m_{cv}}\sum_{i=1}^{m_{cv}} \left( h_{\theta}(x_{cv}^{(i)}) - y_{cv}^{(i)} \right)^2\\
      J_{test}(\theta) &= \frac{1}{2m_{test}}\sum_{i=1}^{m_{test}} \left( h_{\theta}(x_{test}^{(i)}) - y_{test}^{(i)} \right)^2
    \end{align*}
  \item So, how to select $\lambda$? Get $\theta$ from $\min_{\theta}J(\theta)$ with the regularization.
    \begin{center}
      \begin{tabular}{ l c c l l }
        $1) \quad$ & $\lambda=0 \rightarrow$ & $\min_{\theta}J(\theta) \rightarrow $ & $\theta^{(1)} \rightarrow$ & $J_{cv}(\theta^{(1)})$ \\
        $2) \quad$ & $\lambda=0.02 \rightarrow$ & $\min_{\theta}J(\theta) \rightarrow$ & $\theta^{(2)} \rightarrow$ & $J_{cv}(\theta^{(2)})$ \\
        $\vdots$ & & & & \\
        $n) \quad$ & $\lambda=10 \rightarrow$ & $\min_{\theta}J(\theta) \rightarrow$ & $\theta^{(n)} \rightarrow$ & $J_{cv}(\theta^{(n)})$
      \end{tabular}
    \end{center}
    Pick a model that gives the lowest sum of squares $J_{cv}$, e.g. $\theta^{(5)}$. In other words, we fit $\lambda$ to the c.v. set. Then, see how well $J_{test}(\theta^{(5)})$ is.
  \item See lecture 10, slide 23 for a plot of $J_{train}, J_{cv}$ vs. $\lambda$.
\end{itemize}
\hfill \\
{\large \textbf{16-Oct-2020} Learning Curves, Deciding What To Do Next}
\begin{itemize}
  \item Learning Curves; Plot $J_{cv}$/$J_{train}$ as a function of number of training
    examples, $m$.ARtificially reduce the training set to $m$
    examples, for a fixed polynomial degree $h_{\theta}(x)$,
    e.g. $h_{\theta}(x) = \theta_0 + \theta_1x + \theta_2x^2$.
  \item High bias: See lec. 10, p.26:
    $h_{\theta}(x)=\theta_0+\theta_1x$ doesn't fit the data well. More
    data won't help the straight line fit any better. Adding training
    examples only improes $J_{cv}$ up to a certain point. We find $J_{train}$ approaches $J_{cv}$.\\
    $\implies$ if a learning algorithm has high bias, more training e.g.s won't help!
  \item High variance: see lec.10, p.27: e.g. $h_{\theta}(x) =
    \sum_{k=0}^{100}\theta_k x^k$. For small $m$, it'll fit the data
    very well (overfitting). So $J_{train}$ is small for small
    $m$. But as we include more training examples (larger than the
    polynomial degree), $h_{\theta}(x)$ doesn't generalize well, and
    $J_{train}$ increases. $J_{cv}$ starts out high (with a few
    training examples, $h_{\theta}(x)$ generalizes very poorly. With
    more training e.g.s, $J_{cv}$ decreases. \\
    $\implies$ if a learning algorithm has high variance, more training e.g.s will help!
  \item Deciding What To Do Next Revisited; Recall that we're fitting house prices, but the hypothesis made bad predictions.
    \begin{enumerate}
      \item More training e.g.s $\rightarrow$ fixes high variance
      \item Smaller feature set $\rightarrow$ fixes high variance
      \item More features $\rightarrow$ fixes high bias
      \item Add polynomial features $\rightarrow$ fixes high bias
      \item Decrease $\lambda \rightarrow$ fixes high bias
      \item Increase $\lambda \rightarrow$ fixes high varaince
    \end{enumerate}
  \item Neural Networks: Small networks (few units, few hidden layers)
    are prone to underfitting, but computationally cheaper\\ Large
    networks (more units, more hidden layers) are prone to
    overfitting, and computationally more expensive. \\ You can use
    regularization to address overfitting.
  \item How many hidden layers? A sensible default is one hidden layer. But you can use more layers and check $J_{cv}$ and $J_{test}$.
  \item Quiz: n.n. with one hidden layer and $J_{cv}(\theta) \gg
    J_{train}(\theta)$. Will increasing the number of hidden units
    help? No. Since there is a gap between cross validation and
    training sets, it indicates high variance. So it's overfitting the
    data, and adding more layers/units won't help.
\end{itemize}
\hfill \\
{\large \textbf{20-Oct-2020} Prioritizing What To Work On, Error Analysis, Error Metrics for Skewed Classes, Trading Off Precision/Recall, Data for Machine Learning, Large Margin Classification: Optimization Objective, Large Margin Intuition}
\begin{itemize}
  \item Prioritizing What to Work On; Building a spam classifier. Let
    $y=0 \implies$ not spam, and $y=1 \implies$ spam. Features? Choose
    100 words, e.g. $\underbrace{\text{deal, buy, discount,
    }}_{\text{spam words}} \underbrace{\text{andrew, now,
          home}}_{\text{non-spam words}}$. Then, for an e-mail, build a feature vector
    \begin{equation*}
      \begin{matrix} x_1 \\ x_2 \\ \vdots \\ x_{100} \end{matrix} 
      \begin{bmatrix}1 \\ 0 \\ \vdots \\ 1 \end{bmatrix}
      \begin{matrix} \rightarrow \\ \rightarrow \\ \vdots \\ \rightarrow \end{matrix} 
      \begin{matrix}\text{deal} \\ \text{buy} \\ \vdots \\ \text{now}\end{matrix}
    \end{equation*}
    where
    \begin{equation*}
      x_j = \begin{cases} 1 & \mbox{if the word } j \mbox{ is in the e-mail} \\ 0 & \mbox{otherwise }\end{cases}
    \end{equation*}
  \item How to get low error? More data, possibly. Could use more
    sophisticated features based on the header (e.g. routing
    info). Could also use more sophisticated features based on words
    (e.g. deal vs. Dealer; are these the same words?). Could also use
    more features based on punctuation. Could introduce features based on misspelling.
  \item Error Analysis; Look at examples the algorithm gets wrong. Look at the features. What additional features would help improve the accuracy?
  \item Numerical error: Have a single real number as a metric for how good the algorithm is. One example is $J_{cv}$.
  \item See ``porter stemming'' for natural language processing (NLP).
  \item \textit{Always} start with a quick and dirty implementation of an algorithm. Then check learning curves, and then analyze features.
  \item Error Metrics for Skewed Classes; Consider classifying
    benign/malignant ($y=0$/$y=1$) tumors where 99.5\% of the cases
    are benign. Say we've trained a logistic regression, and we get
    99.0\% accuracy. But, suppose our algorithm just always retuend
    $y=0$ regardless of any features $x$. Then, we'd get 0.5\% errror!
    The reason is that this is a skewed class; there is much more of
    one class than the other. For skewed classes, just classification error isn't a good metric.
  \item Precision/Recall:
    \begin{center}
      \begin{tabular}{ | c | c | c| }
        \hline
        \diagbox{Predicted}{Actual} & 1 & 0 \\
        \hline
        1 & True Positive & False Positive \\
        \hline
        0 & False Negative & True Negative \\
        \hline
      \end{tabular}
    \end{center}
    \begin{align*}
      \text{Precision} &= \frac{\text{True Positives}}{\text{Predicted Positives}} \\
      &= \frac{\text{True Positives}}{\text{True Positives + False Positives}} \quad \text{denom is sum of first row in table}\\
      \text{Recall} &= \frac{\text{True Positives}}{\text{Actual Positives}} \\
      &= \frac{\text{True Positives}}{\text{True Positives + False Negatives}} \quad \text{denom is sum of first column in table}
    \end{align*}
  \item So, the classifier which just sets $y=0$ has a recall of zero, since there are no true positives! Having high precision and recall are good metrics.
  \item Let $y=1$ be the rarer class when doing precision/recall.
  \item Trading Off Preicsion/Recall; Continue with the cancer
    example. Suppose we want to report $y=1$ only if we're very
    positive. We could increase the threshold: $h_{\theta}(x) \geq 0.8
    \implies$ higher precision (reduces the false positives) but gives
    lower recall.
  \item Suppose we want to avoid missing malignant cases. Reduce the threshold e.g. $h_{\theta}(x) \geq 0.4 \implies$ lower precision and higher recall.
  \item See lecture 11, p. 13 to see the tradeoff graph between Precision/Recall. (Very loosely, $P = 1-R$.)
  \item How do we choose the threshold? The problem is that $P$ and
    $R$ are two numbers. Suppose you have two algorithms. Algorithm 1
    gives $P=0.2,R=0.7$ and algorithm 2 gives $P=0.8, R=0.1$. Which
    one is better? It's better to have a single real number.
  \item $F_1$ Score:
    \begin{equation*}
      F_1 = \frac{2PR}{P+R}
    \end{equation*}
  \item Data for Machine Learning; For some algorithms, more data can really improve accuracy.
  \item Large data rationale: Assume that features $x \in \mathcal{R}^{n+1}$ can predict $y$. \\
    e.g. ``I ate \_\_\_\_\_\_\_\_\_ eggs for breakfast.'' Should the blank be too/two/to? The features $\{$ate, eggs, breakfast$\}$ are enough to tell us. \\
    e.g. Just the square footage of a house; this is not enough to predict a house price. Need more info such as location, bedrooms, etc. \\
    A litmus test is ``can a human predict $y$ given $x$?''
  \item Now a learning algorithm with many parameters (low bias) will
    fit a training set well. With a very, very large training set,
    unlikely to overfit. The low bias $\implies J_{train} \approx
    0$. ``Unlikely to overfit'' $\implies J_{train} \approx J_{cv}
    \approx 0$.
  \item Large Margin Classification - Optimization Objective; Recall with logistic regreesion
    \begin{align*}
      h_{\theta}(x) &= \frac{1}{1 + \exp \left( - \theta^T x\right)} \\
      z &= \theta^T x \\
      g(z) &= \left(1 + e^{-z}\right)^{-1}.
    \end{align*}
    We know if $z \gg 0 \implies g(z) \approx 1$ and that $z \ll 0 \implies g(z) \approx 0$. Cost (mod a factor of $1/m$):
    \begin{equation*}
      \text{cost} = -y \log h_{\theta}(x) - (1-y) \log( 1 - h_{\theta}(x))
    \end{equation*}
  \item See lecture 12, p. 3 for plots of the $\log(\dots)$. Then,
    come up with a piecewise linear approximation of these
    functions. Call them $\text{cost}_1$ and $\text{cost}_0$,
    corresponding to the $y=1,0$ terms.
  \item For logistic regression, our objective was
    \begin{equation*}
      \min_{\theta} \frac{1}{m} \sum_{i=1}^m \left[ y^{(i)} \left( -\log h_{\theta}(x^{(i)}) \right) + (1 - y^{(i)}) \left( -\log \left(1 - h_{\theta}(x^{(i)})\right)\right)\right] + \frac{\lambda}{2m} \sum_{j=1}^m \theta_j^2
    \end{equation*}
  \item For support vector machines (SVM), drop $1/m$ by convention
    \begin{equation*}
      \min_{\theta}  \sum_{i=1}^m \left( y^{(i)} \text{cost}_1(\theta^Tx^{(i)}) + (1 - y^{(i)}) \text{cost}_0(\theta^Tx^{(i)}) \right) + \frac{\lambda}{2} \sum_{j=1}^m \theta_j^2
    \end{equation*}
  \item We parameterize log. regression as
    $\underbrace{A}_{\text{cost}} + \lambda
    \underbrace{B}_{\text{regression}}$. For SVM, parameterize as $CA
    + B$ where $C$ is the regularization parameter. All regularization
    does is control the relative size of the different terms. Roughly
    speaking, $C=1/\lambda$. So in SVM, we're doing
    \begin{equation*}
      \min_{\theta}  C\sum_{i=1}^m \left( y^{(i)} \text{cost}_1(\theta^Tx^{(i)}) + (1 - y^{(i)}) \text{cost}_0(\theta^Tx^{(i)}) \right) + \frac{\lambda}{2} \sum_{j=1}^m \theta_j^2
    \end{equation*}
  \item SVM's hypothesis
    \begin{equation*}
      h_{\theta}(x) = \begin{cases} 1 & \mbox{if } \theta^Tx \geq 0  \\ 0 & \mbox{otherwise }\end{cases}
    \end{equation*}
  \item Large Margin Intuition; See lecture 12, slide 7 for a summary of SVM cost function and hypothesis.
  \item Suppose we pick a huge $C$. Then we'll want $\sum_{i=1}^m \left( y^{(i)} \text{cost}_1(\theta^Tx^{(i)}) + (1 - y^{(i)}) \text{cost}_0(\theta^Tx^{(i)}) \right)=0$. \\
    So whenever $y^{(i)}=1$, need $\theta^T x^{(i)} \geq 1$ (changed this threshold from 0 to 1 to make the algorithm more ``sure''). Conversely, whenever $y^{(i)}=0$, need $\theta^Tx^{(i)} \leq -1$.
    Really, go look at the slide.
  \item SVM makes decision boundaries that give the largest margin between the classes, hence Large Margin Classifier.
\end{itemize}
\hfill \\
{\large \textbf{21-Oct-2020} Large Margin Intuition cont'd, Math. behind LMC, Kernel I}
\begin{itemize}
  \item Large Margin Intuition (Continued); Outliers: when $C$ is not too large, SVM can ignore some outliers.
  \item Mathematics Behind Large Margin Classifiers; For a very large
    $C$, we are trying to minimize $\sum_{j=1}^n \theta_j^2$, such
    that $\theta^T x^{(i)} \geq 1$ for $y=1$ and $\theta^T x^{(i)}
    \leq -1$ for $y=0$.
  \item Say $n=2$. Then, our minimization objective is
    \begin{equation*}
      \frac{1}{2}\left( \theta_1^2 + \theta_2^2 \right) = \frac{1}{2} \left| \theta \right|^2,
    \end{equation*}
    where, WLOG, $\theta_0=0$ and $\theta$ is a vector.
  \item See lecture 12, p. 13. We can write $\theta^Tx^{(i)}=p^{(i)}\left| \theta \right|$, where $p^{(i)}$ is the projection of $x^{(i)}$ onto $\theta$. The optimization objective is now
    \begin{equation*}
      \frac{1}{2}\sum_{j=1}^n \theta_j^2 \quad \mbox{ if } \quad p^{(i)}\left| \theta \right|  \begin{cases} \geq 1 & \mbox{ if } y^{(i)}=1 \\ \leq -1 & \mbox{ if } y^{(i)}=0 \end{cases}
    \end{equation*}
  \item $\theta$ is orthogonal to the decision boundary. Algorithm
    chooses decision boundary such that $p^{(i)}$ is larger and so
    that $|\theta|$ is smaller. (Based on the requirement that
    $p^{(i)}\left| \theta \right| \geq/\leq -1/+1$.
  \item We assumed $\theta_0 = 0$. If we allow $\theta_0 \neq 0$, we allow boundaries that don't pass through the origin.
  \item Kernels I; We saw complex decision boundaries based on polynomal features. Let's redefine our features. In $x$-space, define some landmarks $l^{(1)},l^{(2)},l^{(3)}$. Make new features given $x$. \\
    \begin{center}
      \begin{tabular}{c c c}
        $x \rightarrow$ & $f_1 = \text{similarity}(x, l^{(1)})$ & $= \exp \left( \frac{ \left| x - l^{(1)} \right|^2}{2 \sigma^2 }\right)$ \\
                        & $f_2 = \text{similarity}(x, l^{(2)})$ & $= \exp \left( \frac{ \left| x - l^{(2)} \right|^2}{2 \sigma^2 }\right)$ \\
                        & $f_3 = \text{similarity}(x, l^{(3)})$ & $= \exp \left( \frac{ \left| x - l^{(3)} \right|^2}{2 \sigma^2 }\right)$
      \end{tabular}
    \end{center}
  \item We call the similarity the ``kernel.'' Gaussian kernel
    here. Write $k(x, l^{(i)})$. The new features die off
    exponentially the further they are from the landmarks.
\end{itemize}
\hfill \\
{\large 22-Oct-2020: Kernels II, Using an SVM}
\begin{itemize}
  \item Where do we get the landmarks? Put the landmarks at the training examples. Given $(x^{(1)},y^{(1)}), \dots, (x^{(m)},y^{(m)})$, let $l^{(k)}=x^{(k)}$.
  \item Given a training example $x^{(i)}$ \\
    \begin{center}
      \begin{tabular}{c c c}
        $x^{(i)}\rightarrow$ & $f_1^{(i)}$ &= $\text{sim}(x^{(i)},l^{(1)})$ \\
                             & $f_2^{(i)}$ &= $\text{sim}(x^{(i)},l^{(2)})$ \\
        &$\vdots$ & \\
                             & $f_i^{(i)}$ &= $\text{sim}(x^{(i)},l^{(i)})=1$ \\
        &$\vdots$ & \\
                             & $f_m^{(i)}$ &= $\text{sim}(x^{(i)},l^{(m)})$ \\
      \end{tabular}
    \end{center}
    (Hold the training e.g. fixed, and get the similarity with every landmark. We've taken an $n(+1)$ dimensional vector ($x^{(i)}$) and turned it into an $m$-dimensional one ($f^{(i)}$). Note that now with these new $f^{(i)}$, $\theta$ will be $m+1$ dimensional.)
  \item So $x^{(i)} \in \mathcal{R}^{n+1}$ gets mapped to $\begin{bmatrix} f_0^{(i)} \\ \vdots \\ f_n^{(i)}\end{bmatrix}$.
  \item Hypothesis: Given $x$, compute $f \in \mathcal{R}^{m+1}$ \\
    Predict $y=1$ if $\theta^T f \geq 1$. \\
    Train:
    \begin{equation*}
      \min_{\theta} \sum_{i=1}^m \left( y^{(i)} \text{cost}_1(\theta^T f^{(i)}) + (1-y^{(i)}) \text{cost}_0 (\theta^T f^{(i)} )\right) + \frac{1}{2}\sum_{j=1}^n \theta_j^2,
    \end{equation*}
    and this will give $\theta$.
  \item How to choose $C$? \\
    Large $C \implies$ high variance \\
    Small $C \implies$ low variance \\
    Large $\sigma^2 \implies$ High Bias (Low Variance)
    Small $\sigma^2 \implies$ High Variance
  \item Using an SVM; Use \texttt{liblinear} or \texttt{libsvm}. We need to chose $C$ and the kernel. (No kernel is the same as linear kernel).
  \item No kernel means that similarity$(x,l)=1$. This does $y=1$ if $\theta^Tx \geq 0$. Use this if we have a large number of features and few training examples.
  \item Gaussian kernel means we need to choose $\sigma^2$ too. Use this kernel if number of features and/or number of training examples is large. Perform feature scaling first!
  \item Not all kernels are valid. They must satisfy Mercer's theorem. Use linear or Gaussian kernels. Another kernel is a polynomial kernel $k(x,l)=(x^Tl + \text{const})^d$. Intuition: if $x$ and $l$ are similar, $x^Tl$ will be large.
  \item Multiclass: most pacakges have multiclass built in. Otherwise, use one-vs.-all.
  \item Logistic Regression vs. SVM:
    \begin{itemize}
      \item If $n \geq m$, use log. regr. or use SVM without a kernel.
      \item If $n$ is small, and $m$ is intermediate, use SVM with a Gaussian kernel.
      \item If $n$ is small, and $m$ is huge, then (i) create or add features and/or (ii) use SVM without a kernel or (iii) use logistic regression.
    \end{itemize}
  \item SVMs have a convex cost function, so they will always find the global minimum.
\end{itemize}
\hfill \\
{\large \textbf{26-Oct-2020}: Clustering, K-Means Algorithm}
\begin{itemize}
  \item Unsupervised Learning: You have a training set $\{ x^{(1)},\dots,x^{(m)}\}$ but no $y$'s. The algorithm will ``find paterns in the data.'' This is clustering.
  \item K-Means Algorithm; This is the most common clustering algorithm. Cf. lecture 13, pp. 6-14.
    \begin{itemize}
      \item Randomly initialize two centroids.  Call them 1 and 2.
      \item Go through all the points and find out if they're closer to centroid 1 or centroid 2. 
      \item For all the points that were labelled as centroid 1, find the ``centre of mass.'' Do the same thing for centroid 2.
      \item Replace the cluster centroids at the centres-of-mass.
      \item Repeat the previous four steps util we're done. We're done when the centroids don't move anymore.
    \end{itemize}
  \item The input to this algorithm is $K$, the number of clusters, the training set $\{ x^{(1)},\dots,x^{(m)}\}$, $x^{(i)} \in \mathcal{R}^n$ (drop $x_0=1$). Randomly assign the centroids $\mu_1, \dots, \mu_K$. \\
    \texttt{for i=1:m}
    \begin{itemize}
      \item $c(i):=$ index of centroid that is closest to $x^{(i)}$, i.e. $c(i) = \min_k \left| x^{(i)} - \mu_k \right|^2$.
    \end{itemize}
    \texttt{for k=1:K}
    \begin{itemize}
      \item $\mu_k:=$ the average of points assigned to cluster $k$. \\ (e.g. if $c^{(1)}=2, c^{(5)}=2, c^{(6)}=2$, then $\mu_k = (x^{(1)} + x^{(5)} + x^{(6)})/3$. \\
        (Note: if there are no points assigned to one centroid, eliminate it [most common], or re-randomize it.)
    \end{itemize}
  \item What if the points are not well separated? E.g. suppose you
    have (weight,height) pairs, and you're trying to figure out for
    which $(w,h)$ should correspond S,M,L t-shirt sizes. Apparently,
    K-means will divide up the data and figure out what $(w,h)$
    correspond to S,M,L. (Unclear about this point.)
\end{itemize}
\hfill \\
{\large \textbf{27-Oct-2020}: Optimization Objective, Random Initialization, Choosing the Number of Clusters, Motivation I: Data Compression, Motivation II: Data Visualization}
\begin{itemize}
  \item Optimization Objective; Introduce $\mu_{c^{(i)}}=$ cluster centroid \textit{of cluster} to which $x^{(i)}$ has been assigned. e.g. $x^{(i)}$ belongs to cluster 5. Then $c^{(i)}=5$ and $\mu_{c^{(i)}}=\mu_5$.
  \item The cost function
    \begin{equation*}
      J(c^{(1)},\dots,c^{(m)},\mu_1,\dots,\mu_k) = \frac{1}{m} \sum_{i=1}^m \left| x^{(i)} - \mu_{c^{(i)}} \right|^2,
    \end{equation*}
    and the optimization objective
    \begin{equation*}
      \min_{\substack{c^{(1)},\dots,c^{(m)} \\ \mu_1,\dots,\mu_k}} J\left(c^{(1)},\dots,c^{(m)},\mu_1,\dots,\mu_k\right).
    \end{equation*}
  \item In our algorithm the first step ($c^{(i)}=$ index of cluster
    centroid closest to $x^{(i)}$, i.e. cluster assignment step) is
    minimizing $J$ w.r.t. $c^{(1)},\dots,c^{(m)}$ while holding $k$
    fixed. \\ The ``move centroid step'' ($\mu_k:=$ mean of points
    assigned to cluster $k$) minimizes $J$ w.r.t. $\mu_1,\dots,\mu_k$.
  \item Random Initialization; Let $K<m$. Randomly assign the centroids to $K$ training examples.
  \item The optimization can run into local optima if initialization
    is unlucky (e.g. if two centroids get randomly picked within the
    same cluster.) To avoid this, use many random initializations and
    run K-means many times.
  \item Run K-means many times (e.g. 50-1000). For each iteration, you'll get a $J\left(c^{(1)},\dots,c^{(m)},\mu_1,\dots,\mu_k\right)$. Choose the iteration that gives the lowest cost.
  \item If you have few clusters e.g. $K$=2-10, then multiple random initiations can be quite helpful. This is less helpful for large $K \sim 10^2 \text{ to } 10^3$.
  \item Choosing the Number of Clusters; Elbow: plot $J$ vs. $k$. If there is a clear ``kink'' or elbow, choose the number of clusters at the kink. Though, this doesn't always happen.
  \item Can also consider the purpose: e.g. S,M,L or XS,S,M,L,XL t-shirts. Three or five clusters.
  \item Motivation I: Data Compression; Highly redundant features,
    e.g. $x_1=$ length in inches, $x_2=$ length in cm. On an
    $x_2$-$x_1$ diagram, they lie on a straight line. $x^{(i)} \in
    \mathcal{R}^2 \rightarrow z^{(i)} \in \mathcal{R}$.
  \item Motivation II: Data Visualization; E.g. Countries have GDP, population, HDI, etc. Maybe $x^{(i)}\in\mathcal{R}^{100}$. If we can reduce this to two features, then we can plot it as $(z_1,z_2)$.
  \item Note that when we reduce dimensionality, the algorithm doesn't
    ascribe any meaning to the new features; that's up to us.
\end{itemize}
\hfill \\ 
{\large \textbf{28-Oct-2020}: Principal Component Analysis, PCA Algorithm, Reconstruction from Compressed Representation}
\begin{itemize}
  \item Principal Component Analysis; Project the data onto a lower dimensional surface such that
    the ``projection error'' (i.e. sum of squares of the distance
    between the point and the surface) is minimal. E.g. For 2-D data,
    find a line (the lower dim. surface) s.t. the sum of squares of
    the distance between the point and its projection onto the line is
    minimal.) See lecture 14, p.11.
  \item PCA problem: \\
    To reduce 2-D to 1-D: find a direction $u^{(1)} \in \mathcal{R}^2$ onto which to project the data so as to minimize the projection errors.\\
    To reduce $n$-D to $k$-D: find $k$ vectors $u^{(1)},\dots,u^{(k)} \in \mathcal{R}^n$ onto which to project the data so as to minimize the projection errors.
  \item PCA is not linear regression; linear regression minimizes a different square distance rror. In PCA, we have no special variable $y$.
  \item PCA Algorithm; Pre-process the data with feature scaling!!!
  \item PCA needs to (i) calculate the $u^{(j)}$ for $j=1,\dots,k$ vectors and (ii) calculate $z^{(i)}$ from $x^{(i)}$.
  \item Algorithm: reduce data from $n$-D to $k$-D. \\
    Compute covariance matrix: $\Sigma = \frac{1}{m}\sum_{i=1}^m x^{(i)}\left( x^{(i)} \right)^T$ \\
    Compute eigenvectors: \texttt{ [U,S,V] = svd}$\left( \Sigma \right)$. \texttt{U} is $n \times n$ with each of the column representing an eigenvector\\
    Take the first $k$ columns (i.e. the first $k$ eigenvectors) to get $U_{red} = \begin{bmatrix} \vline &  & \vline \\ u^{(1)} & \dots & u^{(k)} \\ \vline &  & \vline \end{bmatrix}$ \\
    Then, to get the $k$-dimensional feature $z$, go $z = U_{red}^T x$.
  \item This takes $x \in \mathcal{R}^n \rightarrow z \in \mathcal{R}^k$.
  \item Reconstruction from Compressed Representation; $x_{approx} = U_{red} z$.
\end{itemize}
\hfill \\
{\large \textbf{29-Oct-2020}: Choosing the Number of Principal Components, Advice for Applying PCA}
\begin{itemize}
  \item $k$ is the number of principal components \\
    Average squared projection error: $\frac{1}{m}\sum_{i=1}^m \left| x^{(i)} - x_{approx}^{(i)}\right|^2$ \\
    Total variance of data: $\frac{1}{m}\sum_{i=1}^m \left| x^{(i)} \right|^2$.
  \item Choose $k$ to be the smallest value such that
    \begin{equation*}
      \frac{\sum_{i=1}^m \left| x^{(i)} - x_{approx}^{(i)} \right|^2}{\sum_{i=1}^m \left| x^{(i)} \right|^2} \leq 0.01,
    \end{equation*}
    i.e. 99\% of the variance is retained.
  \item One algorithm (don't use this) would be as follows:
    \begin{itemize}
      \item Try PCA with $k=1$
      \item Computer $U_{red}, z^{(1)}, \dots, z^{(m)}, x_{approx}^{(1)},\dots,x_{approx}^{(m)}$.
      \item Check if $\frac{\sum_{i=1}^m \left| x^{(i)} - x_{approx}^{(i)} \right|^2}{\sum_{i=1}^m \left| x^{(i)} \right|^2} \leq 0.01$.
      \item If it's not, try $k=2$, and so on.
    \end{itemize}
  \item However, this is very inefficient. There is a better way. We did \texttt{[U,S,V] = svd(Sigma)}. The $S$ matrix (covariance matrix) is $\begin{bmatrix}s_{11} & & \\ & \ddots & \\ & & s_{nn}\end{bmatrix}$. It turns out that:
    \begin{equation*}
      \frac{\sum_{i=1}^m \left| x^{(i)} - x_{approx}^{(i)} \right|^2}{\sum_{i=1}^m \left| x^{(i)} \right|^2} = \frac{1 - \sum_{i=1}^k S_{ii}}{\sum_{i=1}^n S_{ii}}.
    \end{equation*}
    \item So, we just need to look at the values of $S$, and keep summing up the diagonal elements until we retain 99\% of the variance.
    \item Advice for Applying PCA; PCA can be used to speed up supervised learning problems. Train on compressed data.
    \item Training set $\{ (x^{(1)},y^{(1)}), \dots, (x^{(m)},y^{(m)}) \}$. Say that $x^{(i)} \in \mathcal{R}^{10,000}$. \\
      Now, $x^{(1)}, \dots, x^{(m)} \xrightarrow[\text{PCA}]{} z^{(1)}, \dots, z^{(m)} \in \mathcal{R}^{1,000}$. \\
      This gives a new training set $\{ (z^{(1)},y^{(1)}), \dots, (z^{(m)},y^{(m)}) \}$\\
      Now use this training set to train the learning algorithm.
    \item PCA gives a mapping from $x \mapsto z$. The mapping should be defined only by running PCA on the training set. \textit{Apply} the mapping to $x_{cv}$ and $x_{test}$.
    \item For visualization, choose $k=2$ or $k=3$.
    \item Might think PCA can be used to avoid overfitting. In other words, reduce the number of features from $n=10,000$ to $k=1,000$. You might think that fewer features means it's less likely to overfit. This is bad! Just use regularization.
    \item Don't automatically jump to PCA; try training learning algorithms on raw data. Only if the algorithm runs too slow, or you run out of disk space, then apply PCA.
\end{itemize}
\end{document}
